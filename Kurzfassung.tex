\chapter*{Kurzfassung}
% Abstract
% what did i do in a nutshell
% wie wenn man kurz einem kollegen die arbeit erklärt
% 200-300 words

%TODO schwäche es wird immer von Umgebung gesprichen mehrdeutig -> besser simulation und darstellung von Daten/Verläufen/Paramter/

%\paragraph*{Motivation}
% why do we care about the problem
%auf entwicklung von mechatronischen systemen eingehen/ heutige Anwendungen sind interdisziplinär 

%Die Entwicklung von mechatronischen Systemen ist eine anspruchsvoller Prozess.
Ein mechatronisches System, wie ein Roboter, ist ein komplexes Zusammenspiel von Teilsysteme, die aus den verschiedenen Disziplinen Mechanik, Elektronik und Informatik stammen.

%Technische Produkte sind heute interdisziplinär. Dadurch bestehen die Produkte aus mehren Teilsystemen. 
%Diese Teilsysteme wiederum  stammen aus den Disziplinen: Mechanik, Elektronik und Informatik.

%Die Entwicklung von mechatronischen Systemen ist schwierig.
%Denn das System setzt sich aus mehren 

%\paragraph*{Problem statement}
% what problem are you trying to solve
%- testen und auswerten von Teilsystemen.
%- erster Entwurf machen

Dieser Umstand macht es schwierig die einzelnen Teilsysteme zu entwickeln und testen.
Um die Entwicklung zu unterstützen und beschleunigen werden Simulations"= und Visualiserungs"=Werkzeuge eingesetzt.
In dieser Arbeit sollen die Möglichkeiten vom \textit{ROS}"=Ökosystem als unterstützendes Werkzeug für \textit{EEROS}"=Applikationen evaluiert werden.
%In dieser Arbeit wird vor allem auf das Entwickeln von Software für mechatronische Systeme eingegangen. 
Im speziellen sollen Lösungen und Vorlagen für die Entwickler von \textit{EEROS}"=Applikationen erarbeitet werden.

%\paragraph*{Approach}
% How did you go about solving
%Umgebung ros -> tools von ros: rviz, gazebo, plots

Dafür werden zwei Fallbeispiele umgesetzt.
Die Umsetzung besteht zum einen aus einer Simulation und zum anderen aus einer Visualisierungs-Lösung.
Mit der Visualisierung sollen Daten und Zustände gleichermassen vom realen Roboter oder vom simulierten Roboter dargestellt werden.

%\paragraph*{Results}
%what is the answer
%aufzeigen an einem einfachen bsp für grundlagen 
%2. bsp an eeduro-delta roboter (siehe titelbild) um aufzuzeigen was möglich oder wie komplexere System entwickelt werden sollen.

Das erste Fallbeispiel ist eine einfachen Motor-Baugruppe.
Mit diesem Beispiel konnte erfolgreich aufgezeigt werden, wie die Entwicklung eines Reglers in \textit{EEROS} unterstützt werden kann.
Ebenfalls wurde eine Entwicklungsumgebung mit Simulation und Visualisierung für den \textit{EEDURO"=Delta} Roboter erstellt.
Diese kann dann eingesetzt werden bei der Entwicklung der  \textit{EEROS}"=Applikation für den Delta-Roboter.
Auch kann mit der Entwicklungsumgebung vom \textit{EEDURO"=Delta} Roboter das Framework \textit{EEROS} kennen gelernt werden ohne zuerst Hardware anzuschaffen.

%\paragraph*{Conclusion}
% folgen von antwort
%entwickler haben nun vorlage/schablone mit der sie arbeiten können

Mit den beiden in dieser Arbeit gezeigten Fallbeispielen erhalten \textit{ROS}"= und \textit{EEROS}"=Entwickler eine Vorlage für das Erstellen von Simulationen und Visualisierungen von Robotern.

%\paragraph*{Stichworte}
%\begin{itemize}
%\item entwickler werkzeuge
%\item Grundlagen mit einfachem bsp
%\item aufzeigen von fähigkeiten an komplexem bsp
%\item gazebo, rivz \& rqt erwähnen
%\end{itemize}