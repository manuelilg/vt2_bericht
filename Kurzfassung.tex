\chapter*{Kurzfassung}
% Abstract
% what did i doe in a nutshell
% wie wenn man kurz einem kollegen die arbeit erklärt
% 200-300 words

%TODO schwäche es wird immer von Umgebung gesprichen mehrdeutig -> besser simulation und darstellung von Daten/Verläufen/Paramter/

\paragraph*{Motivation}
% why do we care about the problem
auf entwicklung von mechatronischen systemen eingehen/ heutige Anwendungen sind interdisziplinär 

Technische Produkte sind heute interdisziplinär. Dadurch bestehen die Produkte aus mehren Teilsystemen. 
Diese Teilsysteme wiederum  stammen aus den Disziplinen: Mechanik, Elektronik und Informatik.

%Die Entwicklung von mechatronischen Systemen ist schwierig.
%Denn das System setzt sich aus mehren 

\paragraph*{Problem statement}
% what problem are you trying to solve
- testen und auswerten von Teilsystemen.
- erster Entwurf machen

Dieser Umstand macht es schwierig die einzelnen Teilsystemen zu entwickeln und testen.
Um die Entwicklung zu unterstützen und beschleunigen sollen Tools eingesetzt werden.
In dieser Arbeit sollen die Möglichkeiten von der ROS und deren Umgebung(frameworkd) als unterstützendes Entwickler-Tool evaluiert werden.
%In dieser Arbeit wird vor allem auf das Entwickeln von Software für mechatronische Systeme eingegangen. 
Im speziellen soll eine Lösung für die Entwickler von EEROS-Applikationen erarbeitet werden.

\paragraph*{Approach}
% How did you go about solving
Umgebung ros -> tools von ros: rviz, gazebo, plots



\paragraph*{Results}
%what is the answer
aufzeigen an einem einfachen bsp für grundlagen 
2. bsp an eeduro-delta roboter (siehe titelbild) um aufzuzeigen was möglich oder wie komplexere System entwickelt werden sollen.

Für das Fallbeispiel mit dem einfachen Motor konnte erfolgreich aufgezeigt werden, wie die Entwicklung für ein EEROS-Applikation unterstützt werden kann.
Ebenfalls wurde eine Entwicklungs-Umgebung für den EEDURO-Delta Roboter erstellt.
Diese kann dann eingesetzt werden bei der Integration der bestehenden EEROS-Applikation für den Roboter in die neuste Version von EEROS.

\paragraph*{Conclusion}
% folgen von antwort
entwickler haben nun vorlage/schablone mit der sie arbeiten können

Mit dieser Arbeit erhalten EEROS-Entwickler eine Vorlage für erstellen von Simulationen in Gazebo und das Darstellen der Resultate.

\paragraph*{Stichworte}
\begin{itemize}
\item entwickler werkzeuge
\item Grundlagen mit einfachem bsp
\item aufzeigen von fähigkeiten an komplexem bsp
\end{itemize}