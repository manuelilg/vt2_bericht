\chapter{Ergebnisse, Fazit und Ausblick}
%TODO Ergebnisse vs Ergebnis 

\section{Ergebnisse}
% Präsens !!!!!!!!!!!!!!

Mit \textit{Gazebo} können Simulationen erstellt werden für \textit{EEROS}"=Applikationen.
Die Simulation kann mit dem \textit{EEROS} synchronisiert werden.
Somit können Echtzeit-Systeme simuliert werden.
Mit dem \textit{RViz} und \textit{RQt} können Daten uns Zustände von Robotern visualisiert werden.


Mit Fallbeispiel Motor"=Apparatur wird gezeigt das ein vollständiges Zusammenspiel zwischen \textit{EEROS} und \textit{ROS} umsetzbar ist.

Die Umsetzung des zweiten Fallbeispiel vom EEDURO-Deltaroboter zeigt die Machbarkeit für komplexere Systeme.

Die Modell"=Beschreibungen und Softwarekomponenten die im Rahmen der beiden Fallbeispiel erstellt wurden, können als Vorlage für andere Projekte dienen.
Einzelnen Softwarekomponenten, wie z.B. das \textit{"gazebo\_ros\_joint\_force"} Plugin, können direkt wiederverwendet werden. 


%% Results
%\begin{itemize}
%\item zeigen von vollständigem Zusammenspiel eeros <-> ros
%\item erros regelt simulaiton
%\item zeigen erstellen Simulation komplexer Systeme
%\item Vorlagen und einzeln Komponenten die wiederverwendet werden können
%\end{itemize}

\section{Fazit}
Im Rahmen dieser Arbeit wurde viel Zeit in die Recherche investiert.
Besonders die Recherchen für wie Modell"=Beschreibung erstellt werden sollen und das wie man \textit{Gazebo} Plugins erstellt.

Die Simulation von einem Delta"=Roboter ist die einzig öffentliche.

Auch ist das das ganze Modell des Roboters in einer einzigen Datei beschrieben ist eine  


Durch die Simulationen für die Motor"=Apparatur und den EEDURO"=Delta kann \textit{EEROS} von Interessierten ausprobiert werden ohne Hardwareanforderungen.
\textit{EEROS}"=Applikationen können durch die Synchronisation zwischen \textit{Gazebo} und \textit{EEROS} auch ohne Realtime"=Kernel entwickelt werden.

 
%% Discussion
%\begin{itemize}
%\item erste Delta-Roboter simulation in gazebo
%\item convenient way model"=beschreibung in einem file
%\item möglichkeit EEROS auszuprobieren
%\item ausprobieren ohne realtime kernel
%\item ausprobieren mit eeduro-Delta
%\end{itemize}

\section{Ausblick}

Weitere Arbeiten die im Rahmen dieser Arbeit gemacht werden könnten sind:

Neue \textit{EEROS}"=Applikationen auf Basis der Vorlagen entwickeln.
Dabei könnten einzelne Software"=Komponenten wieder verwendet werden.
Als Beispiel könnte der EEDURO"=7"=Achsen"=Roboter umgesetzt werden.

Die Software"=Komponenten und Bugfixes könnten als Pull"=Requests in die \textit{ROS}"=Repositorys gemerged werden.

Eine Möglichkeit ist das Simulations"=Modell des EEDURO"=Delta zu erweitern.
Zum eine könnte Funktion für das rotieren des Werkzeugs hinzugefügt werden.
Und zum andern könnte ein Plugin für das \textit{Gazebo} geschrieben werden, so dass auch die Funktion des Magnets, der das Werkzeug vom Roboter ist, simuliert werden kann.


%% further work
%\begin{itemize}
%\item wiederverwenden von einzelnen Komponenten wie plugins
%\item EEROS-Applikation für Delta-roboter auf neuste EEROS-Version updaten
%\item erweitern von Delta-Model mit Rotation Werkzeug
%\item 
%\end{itemize}