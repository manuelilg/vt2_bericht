\chapter{Delta}

%TODO Anhand alle Datenblätter für Modell


\subsection{Vereinfachung}
Eine Vereinfachung für das erstellen und unterhalten von den Dateien für die System"=Beschreibung wurde schon im Kapitel xx Konvertierung vorgestellt. %TODO ref
Mit dem Programm \textit{XACRO} kommt eine weitere hinzu. 

\subsubsection{XACRO} %TODO Scriptsprache
\textit{XACRO} ist eine Makro Sprache für XML"=Dateien. %TODO ref url
Mit \textit{XACRO} können kürzere und einfacher zu unterhaltende XML"=Dateien erstellt werden.
Denn mit Hilfe der Makros können Wiederholungen vermieden werden.
Auch kann Dank \textit{XACRO} eine XML"=Datei in mehrere Unterdateien aufgeteilt werden.
Somit kann ein komplexes System logisch in Untersysteme aufgeteilt werden, die dann jeweils in einer eigenen XML"=Datei beschrieben werden. 
%TODO vorteile einsezen in delta
Im Kapitel XX werden die Vorteile von \textit{XACRO} in der \textit{URDF} "=Datei des Delta-Roboters gezeigt.