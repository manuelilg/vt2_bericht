\chapter{EEDURO-Delta-Roboter}

%TODO Anhang alle Datenblätter für Modell

In diesem Kapitel wird das zweite Fallbeispiel mit dem EEDURO-Delta-Roboter vorgestellt.
Der EEDURO-Delta ist ein kleiner Delta-Roboter, der für Schulungszwecke gedacht ist.
Die Modell-Beschreibung des Roboters ist im \textit{URDF}"=Format gehalten.
Dabei wurde gleich vorgegangen wie für die Motor"=Apparatur\footnote{siehe Kapitel \ref{chap:motor}}, deshalb werden in diesem Kapitel nur noch neue Aspekte erläutert.
%TODO bild

Speziell ist aber das zusätzlich noch die Skriptsprache \textit{XACRO}\footnote{siehe Kapitel \ref{chap:xacro}} eingesetzt worden.

\section{Modell}
\label{chap:delta-modell}
Die Modell"=Beschreibung des Deltaroboters wurde %TODO fertig machen und ref auf xacro
Das Modell des Delta-Roboter kann grob in drei gleiche Teile aufgeteilt werden:
\begin{itemize}
\item Rahmen (base)
\item Arm (arm 1-3)
\item Werkzeug (tool)
\end{itemize}

Der Arm selber besteht aus einem Oberarm auch Link 1 genannt und aus einem Unterarm.
Der Unterarm selber besteht selber wider aus:
\begin{itemize}
\item Doppelgabel (Link 2)
\item zwei Stangen (Link 3.1 und Link 3.2)
\item 2.Doppelgabel (Link 4)
\end{itemize}

%TODO schema roboter

Wie beim Motor muss die kinematisch Kette geschlossen werden mit dem \textit{SDF}"=Joint Element (dargestellt in Abbildung \ref{Ab:delta-struktur} mit grünen Ellipsen).

\begin{figure}[ht!]
	\centering
	\includegraphics[width=12cm]{images/delta_struktur.png}
	\caption{kinematische Struktur EEDURO-Delta}
	\label{Ab:delta-struktur}
\end{figure}

\subsection{Parameter für Modell}
In diesem Abschnitt wird darauf eingegangen welche Parameter es braucht für die Erstellung vom Modell.
\begin{itemize}
\item Masse und Trägheitstensor von allen Links
\item Längenmasse wie Abstand zwischen zwei Gelenken
\item Geometrie von allen Links
\end{itemize}

Ein Teil dieser Parameter konnten aus Datenblättern gewonnen werden.
Diese Datenblätter sind im Repository xx zu finden. %TODO ref

\subsubsection{Berechnung Masse und Trägheitstensor}
Für jedes Glied musste die Masse und Rotationsträgheit berechnet werden, damit das Modell des Roboters sich in der Simulation wie der reale Roboter verhält.
Die Glieder bestehen jedoch aus mehreren Bauteilen.
Dieser Umstand verkompliziert die Berechnung sehr.
Auch erschwerend kam hinzu das die CAD"=Daten nur im \textit{STEP}"=Austauschformat vorliegend waren.

Das führt zu folgendem Arbeitsablauf:
\begin{itemize}
\item für jedes Glied/Baugruppe eine CAD"=Datei erstellen
\item für jede CAD"=Datei:
\begin{itemize}
\item für jedes Bauteil folgende Daten exportieren
\begin{itemize}
\item Volumen von Körper
\item Schwerpunkt von Körper
\item Rotationsträgheit"=Tensor von Volumen
\end{itemize}
\item exportierte Daten in Matlab eintragen
\item für jedes Bauteil Dichte in Matlab hinterlegen
\item für jedes Bauteil mit Dichte: Masse und Rotationsträgheit ausrechnen  
\end{itemize}
\item Schwerpunkt und Masse von Baugruppe bestimmen
\item Rotationsträgheit von Baugruppe ausrechnen (Steinerscher Satz für jedes Bauteil anwenden)
\end{itemize}

Die CAD"=Daten und Matlab"=Dateien sind im Repository xx zufinden.

\subsubsection{XACRO}
\label{chap:xacro}
\textit{XACRO} ist eine Makro Sprache für XML"=Dateien. %TODO ref url
Mit \textit{XACRO} können kürzere und einfacher zu unterhaltende XML"=Dateien erstellt werden.
Denn mit Hilfe der Makros können Wiederholungen vermieden werden.
Auch kann Dank \textit{XACRO} eine XML"=Datei in mehrere Unterdateien aufgeteilt werden.
Somit kann ein komplexes System logisch in Untersysteme aufgeteilt werden, die dann jeweils in einer eigenen XML"=Datei beschrieben werden. 
%TODO vorteile einsezen in delta
Im Kapitel XX werden die Vorteile von \textit{XACRO} in der \textit{URDF} "=Datei des Delta-Roboters gezeigt.



\section{EEDURO-Delta Joint State Publisher}
Wie schon im mehrmals erwähnt braucht %TODO aus anderem Bericht nehmen


Somit gibt es folgenden Prozess Graph.



\subsection{Vereinfachung}
Eine Vereinfachung für das erstellen und unterhalten von den Dateien für die System"=Beschreibung wurde schon im Kapitel xx Konvertierung vorgestellt. %TODO ref
Mit dem Programm \textit{XACRO} kommt eine weitere hinzu. 



\section{Installation und Ausführung}
Verweis auf Repository readme