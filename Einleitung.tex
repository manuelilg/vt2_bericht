\chapter{Einleitung}

\section{Anforderungen/Aufgabenstellung}
% Problem

\begin{itemize}
\item ros konzepte kurz vorstellen wie:
	\begin{itemize}
	\item msg
	\item topic
	\item node
	\end{itemize}
\item ros-ökosystem: gazebo, rviz, rqt
\end{itemize}

In dieser Arbeit wird EEROS als externe Software verwendet.
Jedoch soll hier angemerkt werden dass jede Software verwendet werden kann, insofern die ROS-Library in diese integriert werden kann.
Ein Beispiele wie diese Integration aussehen kann ist in der Arbeit %TODO ref arbeit mäsi zusehen / oder auf repo 


\section{ROS}
\textit{ROS} (Robot Operating System) ist ein Software"=Framework für die Programmierung von Roboteranwendungen.
Mit \textit{ROS} werden vor allem übergeordnete Aufgaben in einer Roboteranwendung umgesetzt.
Es besteht aus einem Set von Softwarebibliotheken und Werkzeugen.
Die einzelnen Teile von \textit{ROS} sind organisiert als Packages.
Das \textit{ROS}"=Netzwerk besteht aus Knoten die über Peer"=to"=Peer miteinander kommunizieren.

Für das Verständnis dieser Arbeit ist ein Grundwissen über \textit{ROS} essentiell.
Im Abschnitt xx werden die Begriffe aus \textit{ROS} nochmals kurz aufgefrischt. %TODO ref
Darum wird für \textit{ROS}"=Neulinge empfohlen sich einen Überblick über \textit{ROS} zu verschaffen.
Gute Quellen für dies sind:
\begin{itemize}
\item Core Componets %TODO ref http://www.ros.org/core-components/
\item ROS Wiki
\item Arbeit Andreas Kalberer
\end{itemize}


\subsection{ROS"=Komponenten}
In diesem Kapitel werden die wichtigsten \textit{ROS}"=Komponenten aufgelistet.
Es werden nur die Begriffe der Komponenten kurz vorgestellt.
Die Begriffe werden bewusst nicht ins Deutsche übersetzt damit die Begriffe geläufig sind wenn man die offizielle \textit{ROS}"=Dokumentation konsultiert.

\paragraph*{Master} \mbox{}\\
Der Master ist die Kernkomponente vom \textit{ROS}"=Netzwerk.
Er verwaltet die Kommunikation zwischen den Knoten (Nodes).
Das Kommunikationsverfahren das verwendet wird ist das Publish\,\&\,Subscribe Prinzip. %TODO ref https://en.wikipedia.org/wiki/Publish%E2%80%93subscribe_pattern

\subsubsection{Nodes}
Knoten sind Prozesse in denen Berechnungen oder Treiber ausgeführt werden. 
Somit kann ein Knoten eine Datenverarbeitung, ein Aktor oder ein Sensor darstellen.  

\subsubsection{Messages}
Die Messages sind eine Konvention für den Austausch von Daten.
Darum gibt es für die unterschiedlichen Anforderungen der Kommunikation, verschiedene Nachricht"=Typen.

\subsubsection{Topics}
Der Nachrichtenaustausch mit dem Publish\,\&\,Subscribe Mechanismus ist anonym.
Darum werden die Nachrichten unter einem Thema (Topic) ausgetauscht.

\section{EEROS}
\textit{EEROS} ist eine Roboter"=Framework mit Fokus auf die Ausführung von Echtzeit"=Aufgaben.
Somit ist es geeignet für die Umsetzung von untergeordneten Aufgaben in einer Roboteranwendung, wie das Regeln von einem Motor.
