\chapter{Einleitung}
%\begin{itemize}
%\item ros konzepte kurz vorstellen wie:
%	\begin{itemize}
%	\item msg
%	\item topic
%	\item node
%	\end{itemize}
%\item ros-ökosystem: gazebo, rviz, rqt
%\end{itemize}
\section{Motivation}





\section{Aufgabenstellung}
Das Ziel dieser Arbeit ist zu evaluieren wie \textit{ROS} als unterstützendes Werkzeug für \textit{EEROS}"=Applikationen eingesetzt werden kann.
%TODO sim und viz erwähnen
%TODO synchron mit 
Dafür soll zuerst ein Konzept ausgearbeitet werden, in dem unter anderem geklärt wird wie die Schnittstellen zwischen \textit{ROS} und \textit{EEROS} aussehen.
Das Konzept soll anschliessend für zwei Fallbeispiele umgesetzt werden.
Eine einfache Motor"=Baugruppe dient als erstes Fallbeispiel.
Anhand dieses Beispiels soll die Korrektheit von dem Konzept gezeigt werden.
Ausserdem soll %TODO arbeiten mit EEROS
Das zweite Fallbeispiel ist der EEDURO"=Deltaroboter.
Mit einem Deltaroboter soll gezeigt werden dass das Konzept auch funktionieren mit komplizierteren Robotern.



%TODO weglassen??
In dieser Arbeit wird EEROS als externe Software verwendet.
Jedoch soll hier angemerkt werden dass jede Software verwendet werden kann, insofern die ROS-Library in diese integriert werden kann.
Ein Beispiel wie diese Integration aussehen kann ist, in der Arbeit %TODO ref arbeit mäsi zusehen / oder auf repo 

\section{Begriffe}
In diesem Abschnitt werden in Kürze die Programme und Begriffe vorgestellt  die in dieser Arbeit verwendet werden.

\subsection{ROS}
\textit{ROS} (Robot Operating System) ist ein Software"=Framework für die Programmierung von Roboteranwendungen.
Mit \textit{ROS} werden vor allem übergeordnete Aufgaben in einer Roboteranwendung umgesetzt.
Es besteht aus einem Set von Softwarebibliotheken und Werkzeugen.
Die einzelnen Teile von \textit{ROS} sind organisiert als Packages.
Das \textit{ROS}"=Netzwerk besteht aus Knoten die über Peer"=to"=Peer miteinander kommunizieren.

Für das Verständnis dieser Arbeit ist ein Grundwissen über \textit{ROS} essentiell.
Im Abschnitt \ref{chap:ros-komponenten} werden die Begriffe aus \textit{ROS} nochmals kurz aufgefrischt.
Darum wird \textit{ROS}"=Neulinge empfohlen, sich einen Überblick über \textit{ROS} zu verschaffen.
Gute Quellen für dies sind:
\begin{itemize}
\item Core Componets\footnote{\url{http://www.ros.org/core-components/}} 
\item ROS Wiki\footnote{\url{http://wiki.ros.org/}}
\item Arbeit \textit{\textsc{"}MME Simulationsprojekt: ROS und Gazebo\textbf{"}} \footnote{siehe Anhang xx} %TODO ref
\end{itemize}

\subsubsection{ROS"=Komponenten}
\label{chap:ros-komponenten}
In diesem Kapitel werden die wichtigsten \textit{ROS}"=Komponenten aufgelistet.
Es werden nur die Begriffe der Komponenten kurz vorgestellt.
Die Begriffe werden bewusst nicht ins Deutsche übersetzt damit die Begriffe geläufig sind, wenn man die offizielle \textit{ROS}"=Dokumentation konsultiert.

\paragraph*{Master} \mbox{}\\
Der Master ist die Kernkomponente vom \textit{ROS}"=Netzwerk.
Er verwaltet die Kommunikation zwischen den Knoten (Nodes).
Das Kommunikationsverfahren das verwendet wird ist das Publish\,\&\,Subscribe\footnote{\url{https://en.wikipedia.org/wiki/Publish\%E2\%80\%93subscribe_pattern}}  Prinzip. 

\paragraph*{Nodes} \mbox{}\\
Knoten sind Prozesse in denen Berechnungen oder Treiber ausgeführt werden. 
Somit kann ein Knoten eine Datenverarbeitung, ein Aktor oder ein Sensor darstellen.  

\paragraph*{Messages} \mbox{}\\
Die Messages sind eine Konvention für den Austausch von Daten.
Darum gibt es für die unterschiedlichen Anforderungen der Kommunikation, verschiedene Nachricht"=Typen.

\paragraph*{Topics} \mbox{}\\
Der Nachrichtenaustausch mit dem Publish\,\&\,Subscribe Mechanismus ist anonym.
Darum werden die Nachrichten unter einem Thema (Topic) ausgetauscht.

\subsection{EEROS} \mbox{}\\
\textit{EEROS}\footnote{\url{http://eeros.org}} ist eine Roboter"=Framework mit Fokus auf die Ausführung von Echtzeit"=Aufgaben.
Somit ist es geeignet für die Umsetzung von untergeordneten Aufgaben in einer Roboteranwendung, wie das Regeln von einem Motor.


\section{Zusammenarbeit}
%TODO fertig machen
Bei der Entwicklung der Kommunikation und Synchronisation mit EEROS hat Marcel Gehrig aktiv mit geholfen.
In der Arbeit \textsc{"}Eine ROS Anbindung für EEROS\textsc{"} von Marcel Gehrig wird beschrieben wie \textit{EEROS} in das \textit{ROS}"=Netzwerk eingebunden wird.

