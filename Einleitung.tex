\chapter{Einleitung}

\section{Anforderungen/Aufgabenstellung}
% Problem

\begin{itemize}
\item ros konzepte kurz vorstellen wie:
	\begin{itemize}
	\item msg
	\item topic
	\item node
	\end{itemize}
\item ros-ökosystem: gazebo, rviz, rqt
\end{itemize}

\section{ROS}
Im ROS-Ökosystem gibt es für die unterschiedlichen Aufgabe jeweils Komponenten bzw. Programme um diese 


\section{EEROS}
kurze Einführung, Verweis

In dieser Arbeit wird EEROS als externe Software verwendet.
Jedoch soll hier angemerkt werden dass jede Software verwendet werden kann, insofern die ROS-Library in diese integriert werden kann.
Ein Beispiele wie diese Integration aussehen kann ist in der Arbeit %TODO ref arbeit mäsi zusehen / oder auf repo 