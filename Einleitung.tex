\chapter{Einleitung}
\section{Motivation}
Bei der Entwicklung von \textit{EEROS}"=Applikationen muss immer eine Hardware vorhanden sein für die die Applikation entwickelt wird.
Dies ist vor allem beim Beginn der Entwicklung ein erschwerender Umstand, da bei Fehlern die Hardware beschädigt werden kann.
Auch gibt es keine einfache und praktische Möglichkeit die Grössen und Zustände von einem System ansprechend zu visualisieren.


\section{Aufgabenstellung}
Das Ziel dieser Arbeit ist zu evaluieren, wie \textit{ROS} als unterstützendes Werkzeug für \textit{EEROS}"=Applikationen eingesetzt werden kann.
Dabei soll vor allem ein Werkzeug für Simulationen und eines für Visualisierungen gefunden werden.
Dafür soll zuerst ein Konzept ausgearbeitet werden, in dem unter anderem geklärt wird, wie die Schnittstellen zwischen \textit{ROS} und \textit{EEROS} aussehen.
Das Konzept soll anschliessend für zwei Fallbeispiele umgesetzt werden.
Eine einfache Motor"=Apparatur dient als erstes Fallbeispiel.
Anhand dieses Beispiels wird die Korrektheit des Konzeptes überprüft.
Denn es soll eine \textit{EEROS}"=Applikation mithilfe der im Fallbeispiel erstellten Simulation entwickelt und getestet werden.
Das zweite Fallbeispiel ist der EEDURO"=Deltaroboter.
Mit einem Deltaroboter soll gezeigt werden, dass das Konzept auch mit komplizierteren Robotern funktioniert.

\section{Begriffe}
In diesem Abschnitt werden in Kürze die Programme und Begriffe vorgestellt, die in dieser Arbeit verwendet werden.

\subsection{ROS}
\textit{ROS} (Robot Operating System) ist ein Software"=Framework für die Programmierung von Roboteranwendungen.
Mit \textit{ROS} werden vor allem übergeordnete Aufgaben in einer Roboteranwendung umgesetzt.
Es besteht aus einem Set von Softwarebibliotheken und Werkzeugen.
Die einzelnen Teile von \textit{ROS} sind organisiert als Packages.
Das \textit{ROS}"=Netzwerk besteht aus Knoten die über Peer"=to"=Peer miteinander kommunizieren.

Für das Verständnis dieser Arbeit ist ein Grundwissen über \textit{ROS} essentiell.
Im Abschnitt \ref{chap:ros-komponenten} werden die Begriffe aus \textit{ROS} nochmals kurz aufgefrischt.
Darum wird \textit{ROS}"=Neulingen empfohlen, sich einen Überblick über \textit{ROS} zu verschaffen.
Gute Quellen für dies sind:
\begin{itemize}
\item Core Componets\footnote{\url{http://www.ros.org/core-components/}} 
\item ROS Wiki\footnote{\url{http://wiki.ros.org/}}
\item Arbeit \textit{\textsc{"}MME Simulationsprojekt: ROS und Gazebo\textbf{"}} \footnote{siehe Anhang xx} %TODO ref
\end{itemize}

\subsubsection{ROS"=Komponenten}
\label{chap:ros-komponenten}
In diesem Kapitel werden die wichtigsten \textit{ROS}"=Komponenten aufgelistet.
Es werden nur die Begriffe der Komponenten kurz vorgestellt.
Die Begriffe werden bewusst nicht ins Deutsche übersetzt damit die Begriffe geläufig sind, wenn man die offizielle \textit{ROS}"=Dokumentation konsultiert.

\paragraph*{Master} \mbox{}\\
Der Master ist die Kernkomponente vom \textit{ROS}"=Netzwerk.
Er verwaltet die Kommunikation zwischen den Knoten (Nodes).
Das Kommunikationsverfahren das verwendet wird, ist das Publish\,\&\,Subscribe\footnote{\url{https://en.wikipedia.org/wiki/Publish\%E2\%80\%93subscribe_pattern}}  Prinzip. 

\paragraph*{Nodes} \mbox{}\\
Knoten sind Prozesse in denen Berechnungen oder Treiber ausgeführt werden. 
Somit kann ein Knoten eine Datenverarbeitung, ein Aktor oder ein Sensor darstellen.  

\paragraph*{Messages} \mbox{}\\
Die Messages sind eine Konvention für den Austausch von Daten.
Darum gibt es für die unterschiedlichen Anforderungen der Kommunikation, verschiedene Nachrichten"=Typen.

\paragraph*{Topics} \mbox{}\\
Der Nachrichtenaustausch mit dem Publish\,\&\,Subscribe Mechanismus ist anonym.
Darum werden die Nachrichten unter einem Thema (Topic) ausgetauscht.

\subsection{EEROS} \mbox{}\\
\textit{EEROS}\footnote{\url{http://eeros.org}} ist ein Roboter"=Framework mit Fokus auf die Ausführung von Echtzeit"=Aufgaben.
Somit ist es geeignet für die Umsetzung von low-level Aufgaben in einer Roboteranwendung, wie das Regeln von einem Motor.

\section{Zusammenarbeit}
An der Ausarbeitung des Konzeptes hat Marcel Gehrig aktiv mitgewirkt.
Denn die \textit{ROS}"=Schnittstelle im \textit{EEROS}, die er im Rahmen seiner Arbeit \textit{\textit{"}Eine ROS Anbindung für EEROS\textit{"}}~\footnote{\url{https://github.com/MarcelGehrig2/berichtVt2/blob/FinalRelease/Bericht.pdf}} aus dem Jahr 2017 entwickelt hat, ist eine integraler Bestandteil des Konzepts.
Auch hat er die \textit{EEROS}"=Applikation für das Fallbeispiel Motor"=Apparatur entwickelt und somit die Umsetzung des Fallbeispiels validiert.

\section{Repositories}
Alle in dieser Arbeit erwähnten Repositories sind öffentlich und auf GitHub gehostet.

\begin{tabular}
  { l						l			 												l						}

% Name						URL   														Branch     			
  \textbf{Name}				& \textbf{URL}												& \textbf{Branch}		\\
  motor\_sim				& https://github.com/manuelilg/motor\_sim.git				& master				\\
  eeduro\_delta				& https://github.com/manuelilg/eeduro\_delta.git			& master				\\
  gazebo\_ros\_joint\_force	& https://github.com/manuelilg/gazebo\_ros\_joint\_force.git& master		 		\\
  gazebo\_ros\_pkgs			& https://github.com/manuelilg/gazebo\_ros\_pkgs.git		& master				\\
  Bericht					& https://github.com/manuelilg/vt2\_bericht.git				& master				\\
  EEROS-ROS					& https://github.com/MarcelGehrig2/VT2/tree/master/Software	& master				\\
	
\end{tabular}
