\chapter{Konzept}
% idee vorstellen 
\subsubsection*{Stichworte}
\begin{itemize}
\item Übersicht von Konzept mit Graphik
\item 3 Bereiche aufzeigen: EEROS, Gazebo, rqt \& rviz
\item Schnittstellen zu einander
\item Verweis zu Mäsis Arbeit
\item urdf, sdf vorstellen und erklären, hier??? oder in Motor
\end{itemize}

Das Konzept kann in mehrere Bereich aufgeteilt werden.
Die Simulation des Systems, die Visualisierung des Zustandes vom System und die Schnittstelle zum EEROS.



\section{EEROS}
kurze Einführung, Verweis

In dieser Arbeit wird EEROS als externe Software verwendet.
Jedoch soll hier angemerkt werden dass jede Software verwendet werden kann, insofern die ROS-Library in diese i werden kann.
Ein Beispiele wie diese Integration aussehen kann ist in der Arbeit %TODO ref arbeit mäsi zusehen / oder auf repo 


\section{Gazebo}

\section{Darstellung}
\subsection{rviz}
\subsection{rqt}
gui-tool, für das plugins vorhanden sind oder selber geschrieben werden können
hat für viele gängige ros-cli-tools ein rqt-plugin


\section{Model-Beschreibung}

\subsection{URDF-Unified Robot Description Format}
% problem: wie anfangen: besch. urdf und sdf oder urdf dann erst sdf
urdf vorstellen
%bilder -> https://ni.www.techfak.uni-bielefeld.de/files/URDF-XACRO.pdf

\begin{itemize}
\item XML format
\item kinematic tree structure
\end{itemize}

\subsubsection*{Link}

\subsubsection*{Joint}

