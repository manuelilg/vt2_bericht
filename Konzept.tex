\chapter{Konzept}
% idee vorstellen 
\subsubsection*{Stichworte}
\begin{itemize}
\item Übersicht von Konzept mit Graphik
\item 3 Bereiche aufzeigen: EEROS, Gazebo, rqt \& rviz
\item Schnittstellen zu einander
\item Verweis zu Mäsis Arbeit
\item urdf, sdf vorstellen und erklären, hier??? oder in Motor
\end{itemize}

Eine Skizze des Konzeptes ist in der Abbildung %TODO ref
zusehen.
Die zwei Hauptkomponenten sind \textit{EEROS} und \textit{ROS}.
\textit{EEROS} übernimmt in diesem Szenario die Aufgabe des Regelns.
Und \textit{ROS} die Aufgaben der Simulation und der Visualisierung.

Damit \textit{EEROS} mit \textit{ROS}"=Komponenten kommunizieren kann, wurde einen \textit{ROS}"=Node ins \textit{EEROS} integriert.
Somit hat \textit{EEROS} die Fähigkeit über das \textit{ROS}"=Kommunikationsprinzip \textit{Publish\,\&\,Subscribe} Daten auszutauschen.
Das Konzept sieht vor das \textit{EEROS} wahlweise ein echtes oder simuliertes System regelt.
Und in beiden Fällen können die Zustände und Regel"=Parameter durch \textit{ROS} visualisiert werden.

Für die Simulation von Systemen wird \textit{Gazebo} eingesetzt.
Die Darstellung von den System Zuständen wird mit \textit{rviz} realisiert. Und für die Darstellung von Daten wie der Regel"=Parameter in Graph"=Plots wird \textit{rqt} eingesetzt.

%Das Konzept kann in mehrere Bereich aufgeteilt werden.
%Die Simulation des Systems, die Visualisierung des Zustandes vom System und der KontrollerEEROS.

\section{Gazebo}
\begin{itemize}
\item festkörper simulation
\item kann mit selbst geschriebenen Plugins erweitert werden.
\item benötig sdf für beschreibung des zu simulierenden systems
\end{itemize}

\textit{Gazebo} ist eine Simulationsumgebung für Starrkörper.
Das zu simulierenden Systems wird mit einem \textit{sdf}"=File beschrieben. %TODO ref zu sdf
Das File ist im XML"=Format aufgebaut.

Um \textit{Gazebo} mit Funktionen zu erweitern können Plugins verwendet werden.
Dabei kann man schon fertige Plugins verwenden oder selber eines programmieren. %TODO ref auf fertige plugins

\section{Darstellung}
\subsection{rviz}
\begin{itemize}
\item darstellen von System und deren Zuständen
\item benötigt urdf-file für Darstellung
\end{itemize}
Das Visualisierungs"=Tool \textit{rviz} ist für die Darstellung von Systemen geeignet.
Dabei können z.B. Zustände von einem Roboter visualisiert werden oder Sensor"=Daten wie von einer 3D"=Kamera.

Ein System besteht aus mehren Körpern.
Um sie darzustellen braucht es Informationen über das Aussehen und Form dieser.
Diese Informationen müssen dem \textit{rviz} über eine \textit{urdf}"=File zur Verfügung gestellt werden. %TODO ref urdf
Für die Darstellung der Körper im Raum benötigt \textit{rviz} für jeden Körper die Position und Orientierung von diesem im Raum.
Deshalb müssen dem \textit{rviz} stetig Koordinaten-Daten für jeden Körper übermittelt werden. %TODO ref tf

\subsection{rqt}
\textit{rqt} ist Framework für die GUI Entwicklung in \textit{ROS}.
Diese GUI's werden als Plugins implementiert.
Somit können mehre GUI's in einem \textit{rqt}"=Fenster verwendet werden.
Für die Darstellung von zeit veränderlichen Parametern wir das Plugin \textit{multiplot} verwendet. %TODO ref und bild?

%gui-tool, für das plugins vorhanden sind oder selber geschrieben werden können
%hat für viele gängige ros-cli-tools ein rqt-plugin


\section{Model-Beschreibung}
% oder System



\subsection{URDF-Unified Robot Description Format}
% problem: wie anfangen: besch. urdf und sdf oder urdf dann erst sdf
urdf vorstellen
%bilder -> https://ni.www.techfak.uni-bielefeld.de/files/URDF-XACRO.pdf

\begin{itemize}
\item XML format
\item kinematic tree structure
\end{itemize}

\subsubsection*{Link}

\subsubsection*{Joint}

